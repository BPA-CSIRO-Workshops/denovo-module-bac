% Define the module top matter
% This gets used to create the chapter title page
% NOTES:
%  * When multiple people have authored or contributed to the module, simply use a LaTeX line break
%    (a double-backslash: \\) at the end of each person.
%  * If you don't want this information shown on the module chapter page, simply remove the lines
%    within the \setModuleAuthors{} and \setModuleContributions{} environments
\setModuleTitle{Bacterial Assembly} %Your first genome assembly
\setModuleAuthors{%
  Torsten Seeman \mailto{tseemann@unimelb.edu.au} \\
  Paul Berkman \mailto{Paul.Berkman@csiro.au} \\
  Philippe Moncuquet \mailto{Philippe.Moncuquet@csiro.au} \\
}
\setModuleContributions{%
  Zhiliang Chen \mailto{zhiliang.chen@unsw.edu.au} \\%
  Sonika Tyagi \mailto{Sonika.Tyagi@agrf.org.au}
}

% BEGIN: Module Title Page
% This simply uses the above information and creates a module chapter page
% NOTES:
%  * The chapter page will always appear on odd numbered page
\chapter{\moduleTitle}
\newpage
% END: Module Title Page


% BEGIN: KLOs
% Key Learning Outcomes (KLOs) are an important aspect of any learning/training. They provide
% valuable infomation about what the trainee will have learned, what they will be able to do or know
% abouti at the end of the module. Unlike objectives which are more trainer oriented, KOLs are
% focused on the learner.
% At the end of the module, the KLOs can be used to develop criteria for writing an assessment to
% see if the trainees knowledge/skills have improved as a result of the module.
% 
% Search online for information on how to write KLOs. e.g.
% http://www.teaching-learning.utas.edu.au/__data/assets/word_doc/0014/23333/Learning-outcomes-v9.1.doc
\section{Key Learning Outcomes}

After completing this module the trainee should be able to:
\begin{itemize}
  \item Perform a simple genome assembly for a small organism using \texttt{Velvet}
  \item Visualise the assembly graph using \texttt{Bandage}
  \item Be aware of the effects and trade-offs of the parameter \texttt{K} on the genome assembly
  \teim Understand that genome assembly produces a graph structure
\end{itemize}
% END KLOs

% BEGIN: Resources Used
% This section can be used to describe the tools and data used during the module. It helps to act as
% a future reference to the trainee
\section{Resources You'll be Using}
 
\subsection{Tools Used}
\begin{description}[style=multiline,labelindent=0cm,align=left,leftmargin=0.5cm]
  \item[Velvet]\hfill\\
  	\url{https://www.ebi.ac.uk/~zerbino/velvet/}
  \item[Bandage]\hfill\\
  	\url{https://rrwick.github.io/Bandage/}
 % \item[Picard]\hfill\\
 % 	\url{http://picard.sourceforge.net/}
\end{description}

%\section{Useful Links}
 
%\begin{description}[style=multiline,labelindent=0cm,align=left,leftmargin=0.5cm]
%  \item[FASTQ Encoding]\hfill\\
%    \url{http://en.wikipedia.org/wiki/FASTQ_format#Encoding}
%\end{description}

\newpage
% END: Resources Used

% BEGIN: Introduction
\section{Introduction}

% To make a paragraph appear as a "note" to the reader, simply wrap it in a "note" environment like
% this:
%\begin{note}
In this tutorial we will take raw sequencing reads and \textit{de novo} assemble them
into contigs. We will also explore the internal assembly graph structure to
aid our understand where, how and why assemblies are incomplete.

%\end{note}

\section{Before the assembly}
\subsection{Sequence data}
We have sequenced the genome of an bacterium using paired-end chemistry on 
an Illumina HiSeq 2000 instrument at the University of Oxford and publicly available as SRR2054105\url{http://www.ebi.ac.uk/ena/data/view/SRR2054105}.

\begin{steps}
Let's have a look at the datasets by doing:
\begin{lstlisting}
 cd ~/ %path???
 ls
\end{lstlisting}
 R1.fastq.gz
 R2.fastq.gz
\end{steps}


\begin{questions}
How many reads are in this data set?
\begin{answer}
% Needs answer
\end{answer}
What is the yield in basepairs?
\begin{answer}
% Needs answer
\end{answer}
Assuming an average bacterial genome size of 4 Mbp, what depth of coverage do we have?
\begin{answer}
% Needs answer
\end{answer}

\end{questions}



\subsection{Trim or clip reads}

There are two distinct reasons one may wish to trim or clip the raw reads.
\begin{enumerate}
\item Low quality bases typically occur toward the end of Illumina reads. The lower the quality score, the higher the chance that the base is incorrect. 
This introduces false k-mers into the assembly process. A good assembler should handle these gracefully.
\item Seqeuncing adaptors are artificial sequence that can occur at the end of reads that came from fragments of DNA that were shorter than desired. 
They were not in the original genome and their presence in lots of reads will totally confuse the assembler.
\end{enumerate}

Normally one would use \texttt{fastqc} or similar to determine whether quality trimming is warranted and for the presence of sequencing adaptors. 
The data set in this exercise is 99\% free of adaptors and is good quality overall so we will skip those steps.
\begin{questions}
How do you know which adaptor sequence to trim?
\begin{answer}
% Needs answer
\end{answer}
How are are quality values in \texttt{FASTQ} files calculated? Can we trust them?
\begin{answer}
% Needs answer
\end{answer}
\end{questions}

\subsection{Allocate yourself to a \texttt{K} value on the spreadsheet}

An important parameter to most genome assemblers is \texttt{K} which is the k-mer size used to construct the \textit{de Bruijn} graph (pronounced \textit{de Brown}) .

Please go to this shared online spreadsheet\url{https://docs.google.com/spreadsheets/d/1iFbCCihawpY1LClsAB-OJ66lyeW7EsdJyaMo1HCetF8/edit?usp=sharing} and choose \textbf{a K value} and put your name next to it.


\section{Assemble the reads using Velvet}
The \texttt{Velvet} software performs the assembly in two steps. The first \texttt{velveth} step hashes the reads. The second \texttt{velvetg} step builds, cleans and traverses the graph.

\begin{steps}
Choose an output folder \texttt{DIR} and use the \texttt{K} value you were allocated on the spreadsheet and assemble the reads.
% need to specify DIR
\begin{lstlisting}
velveth DIR K –shortPaired –fastq.gz –separate R1.fastq.gz R2.fastq.gz 
/usr/bin/time –f "\%e" velvetg DIR -exp_cov auto –cov_cutoff auto
\end{lstlisting}
\end{steps}

Using the \texttt{dir/velvet.out} and \texttt{dir/Log} files determine the values for the following assembly statistics and add them to the spreadsheet \url{https://docs.google.com/spreadsheets/d/1iFbCCihawpY1LClsAB-OJ66lyeW7EsdJyaMo1HCetF8/edit?usp=sharing} from earlier:
\begin{table}[H]
  \centering
  \caption{Output files}
    \begin{tabular}{ll}
    \toprule
    \textbf{Filename} & \textbf{Description} \\
    \midrule
    \texttt{contigs.fa} & this is a FASTA file with your contigs \\
    \texttt{Log} & has most of the metrics in it that we recorded \\
	\texttt{stats.txt} & TSV file of length and coverage of individual contigs \\ 
	\texttt{Sequences} & A copy of the input FASTQ sequences in FASTA format \\ 
	\texttt{Pregraph Roadmaps Graph2} & Interim assembly graph structure \\ 
	\texttt{LastGraph} & Final graph structure
    \bottomrule
    \end{tabular}
  \label{tab:velvet_out}
\end{table}

\begin{steps}

Let's examine the \texttt{stats.txt} file and look at the \texttt{short1_cov} column which is the k-mer coverage of each contig:
\begin{lstlising}
cut –f6 dir/stats.txt | less
\end{lstlisting}
\end{steps}

\begin{questions}
What do you notice about the k-mer coverages?
\begin{answer}
% Needs answer
\end{answer}
What do the outliers correspond to?
\begin{answer}
% Needs answer
\end{answer}
\end{questions}

\begin{steps}
\begin{lstlisting}
grep NN dir/contigs.fa
\end{lstlisting}
\end{steps}

\begin{questions}
Why is there N letters in the assembly?
\begin{answer}
% Needs answer
\end{answer}
\end{questions}

\section{Visualising the assembly graph using Bandage}

The final graph that \texttt{Velvet} uses is stored in the file \texttt{LastGraph}. The \texttt{Bandage} software allows us to view and explore the assembly graph.
\begin{steps}
\begin{enumerate}
\item Start \texttt{Bandage}.
\item Go to \texttt{File - > Load Graph} and load the \texttt{LastGraph} from your assembly in \texttt{DIR}. This may take a little while so be patient.
\item Maximize your window to fill up the whole screen.
\item Click \texttt{Draw graph} on the left hand panel.
\item Change \texttt{Random colours} to \texttt{Colour by read depth} on the left hand panel.
\end{enumerate}
\end{steps}

Now get up out of your chair and walks around and look at all the different graphs the other participants obtained with different values of \texttt{K}.
\begin{questions}
How does \texttt{K} affect the graph? \\
\begin{answer}
% Needs answer
\end{answer}
What would the graph look like in an ideal situtation? \\
\begin{answer}
% Needs answer
\end{answer}
Why didn't anyone achieve perfection? \\
\begin{answer}
% Needs answer
\end{answer}
\end{questions}

Here is another example \url{https://github.com/rrwick/Bandage/wiki/Effect-of-kmer-size} from the Bandage web site. % MORE DETAIL???

\subsection{Explore the graph}

\texttt{Bandage} is designed to be an interactive tool. It allows you to see relationships between parts of your genome that are lost when you only look at the FASTA file of contigs.
\begin{itemize}
\item Zoom in using the \texttt{Zoom} up/down button in the left hand panel
\item Pan around by holding down the Option/Windows key and dragging on the background
\item Move nodes out of the way by selecting and dragging
\end{itemize}
Feel free to play around a bit and ask questions.

\subsection{Features of the assembly graph}
The graph is quite tangled. The long stretches correspond to contigs. The intersections correspond to shared k-mers in the graph, which occur due to repeated sequences within the genome.
\begin{steps}
\begin{enumerate}
\item Select an node (rectangle) in the graph. It's length is reported in the right hand panel as \texttt{Length: NNN}.
\item On the menu choose \texttt{Output -> Web BLAST selected node}. Your browser should open the
   NCBI BLAST Website \url{http://blast.ncbi.nlm.nih.gov/Blast.cgi?PROGRAM=blastn&PAGE_TYPE=BlastSearch&LINK_LOC=blasthome}.
\item Click the \texttt{BLAST} button at the bottom, and wait for the result.
\end{steps}

\begin{questions}
Did your node match anything in the Genbank database? \\
\begin{answer}
%Answer
\end{answer}
Can you determine what species of bacteria was sequenced? \\
\begin{answer}
%Answer
\end{answer}
\end{questions}

\section{Conclusion}
You should now:
\begin{itemize}
\item know how to use Velvet to assemble a simpel genome from Illumina sequences
\item understand the role of the k-mer length K in the assembly process
\item be able to relate the graph structure to the final contigs
\item realize the limitations of short read sequences with respect to genome complexity
\end{itemize}
Thank you.
